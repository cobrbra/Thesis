\documentclass[../thesis.tex]{subfiles}

\begin{document}

\begin{abstract}
In this work, we employ a variety of methods, some novel, from high-dimensional statistics and machine learning to high-throughput cancer genomics data. We restrict our enquiry in the following two ways. Firstly, we are concerned with understanding genomic instability and the hypermutated phenotype, particularly in the context of its relevance to immunotherapy. Secondly, while we use a variety of 'omics data types to inform our understanding, our specific aim will be to make clinical predictions on the basis only of whole-genome, whole-exome, or targeted panel sequencing of the tumour genome. This can be motivated via both via scientific interest and clinical practicality. While somatic mutations have been understood for a long time as the instigators and drivers of tumourigenesis, it is now well known that the tumour environment and external factors play key roles in cancer development and spread. It is therefore of key importance to appreciate the nature of this balance. We formulate our interrogation of this broad question as a signal-to-noise problem, and attempt to ascertain to what extent the downstream properties of tumours can be predicted purely on the basis of the mutations they carry. This is in turn directly applicable to a growing area of clinical practice, liquid biopsy. Unlike solid biopsy, the nature of liquid biopsy means that we are only able to sequence tumour DNA, requiring any decisions based on liquid biopsy to be based purely on somatic mutations, potentially for some subset of the genome. We present a blueprint for the design, development and estimation of clinical biomarkers of response to immunotherapy based on generative models of the genome-wide landscape of molecular processes in tumour cells. From such models we can also make inferences about the underlying mechanisms leading to genomic instability, and how the hypermutated phenotype interacts with the body's natural defences against cancer. 
\end{abstract}

\end{document}