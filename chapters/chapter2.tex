\documentclass[../thesis.tex]{subfiles}

\begin{document}

We now describe a workflow that will be a common theme throughout the rest of this work. Namely, we will \textit{i)} identify a biomarker of interest; \textit{ii)} fit a statistical model to the distribution of mutations 

\section{Generative models of mutation}


\subsection{The nature of generative models}
Generative models attempt to capture the underlying distribution of complex data in a manner that allows new samples to be drawn from the same distribution efficiently. They come in a variety of classes and have been a particular focus of research in the machine learning community in the last decade, being utilised for data compression, representation learning, and as a means to generate new samples from complex distributions. Generative models have found particular application in  in disciplines dealing with extremely high-dimensional, complex data distributions, including image analysis and natural language processing as well as, more recently, genomics. Generative models often (but not always) attempt to learn some lower-dimensional latent representation of their high-dimensional inputs, and as such are related to the theory of dimensionality reduction, and of unsupervised learning in general. 



\subsection{Strategies for fitting generative models}


\subsection{Discrete, high-dimensional, sparse and bursty: the challenges of mutation data}

\citep{zhao_variational_2020}

\subsection{Previous work in generative models of genome-wide mutation}
\citep{budczies_cutoff_2012} \citep{yao_ectmb_2020} \citep{fantini_mutsignatures_2020}

\section{Learning from learning: biomarker prediction}
\section{The double role of regularisation}
This section consists, in part, of discussion adapted from 'Dimensionality and Structure in Cancer Genomics: A Statistical Learning Perspective' \citep{bradley_dimensionality_2020}. This was published as the third chapter of the book 'Artificial Intelligence in Oncology Drug Discovery and Development' \citep{cassidy_artificial_2020}.

\dobib % renders bibliography (only when compiling for chapter only)


\end{document}
