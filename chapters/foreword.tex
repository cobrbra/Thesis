\documentclass[../thesis.tex]{subfiles}

\begin{document}

The work contained in this thesis constitutes the result of many partnerships, each of which I have been lucky to have been involved in. These have enabled me to continue working (to some extent) fruitfully and (to a greater extent) happily since starting my PhD at Edinburgh in 2019. In many cases, however, their roots were planted before that, and in even more cases they will hopefully continue to grow long after everyone who will ever read this document has done so\footnote{Current estimates inform me that I should expect this lucky few to number between zero and three (inclusive).}. 

To begin with some self-indulgently biographical specifics, I would not be where I am today were it not for being randomly assigned to work in the Elowitz Laboratory at Caltech on a summer placement in 2017. At that point I had a passing interest in biology, passing enough to pass over (for example) choosing to be educated in it in any way beyond what was legally required. Despite this, somehow Michael Elowitz, Yaron Antebi, and Christina Su were persuaded to babysit me in a professional laboratory for three months, and were in the process responsible for the shape of my life ever since. While I've moved into a different field of biology, I wouldn't have been on the right farm were it not for their inspiration.

Academically, the next leap for me was being encouraged to take on a Master's in Systems Biology by the likes of Anindya Sharma and my own lack of enthusiasm to join the workforce. Still a pretty clueless maths student then as much as now, I owe several people a great deal of thanks for surviving that year, including Pavel Artemov, Stephen Cole, Sean Jones, Dan Mirea, Carolina Monck, Hannah Munby, and Steve Russell.

While this was all going on, some new folks turned up, and they worked for a company called Cambridge Cancer Genomics. Shortly thereafter I also worked (sometimes) for a company called Cambridge Cancer Genomics, and this is where I first really got a sense that what I was doing there I'd be doing for quite a bit of the rest of my life. From Hannah Thompson and Belle Taylor's welcome and support (continuing to this day), to Nirmesh Patel and Harry Clifford's trusting, supportive, and empowering supervision, to Henry Farmery's lessons on the significance of unit testing, to Dobby, Dami, and Geoff's reminders that if I had a long way to go academically I had further to go as a Mario Kart driver, I learned as much during my time there as I've learned at any other time since. Oh, and they offered to fund my PhD.

All good things must come to an end, and while my friendships from CCG have certainly not ended, unfortunately the company did. At this point, I was in the market for some new (academic) partners. Luckily, a measured and thoughtful American called Michael Mayhew wandered across my mid-pandemic Zoom window and started talking to me about whacko Bayesian stuff. It turned out that he had a gaggle of his own (hello Diego Borges, Ljubomir Buturovic, Mafalda Cavaleiro, Arthur Radley, Sara Masarone, Amitesh Pratap, Melissa Remmel, and Yuan Yuan), and what was intended to be a few months of summer fun and a chance to get away from my PhD thesis became (at the time of writing, and counting in both respects) two and a bit years of fun and a pretty substantial portion of my PhD thesis. The gentle mentorship and flexibility shown by all of these folks, and especially Michael, was truly transformative for me as a researcher and person. 

Getting towards the end of this unrequested (and please God unrequited) academic synopsis, I need to get to the people who've been supportive throughout the years that I've actually been doing my PhD (rather than swanning off for months at a time without explanation or, arguably, justification). Firstly the Usual Suspects: Andrew Beckett, Jamie Burke, Bella Deutsch, Linden Disney-Hogg, Jon Eugster, Josh Fogg, Augi Jacovskis (got your surname right first try), Mary Llewellyn (not so lucky). You are amongst you: my longest and most co-dependent flatmate, my principal collaborator in projects that have borne no revelance to my PhD, my favourite, and five other of my friends. I will allow you to fight it out between you who is who. Next my research group, which has grown infuriatingly from just me to me and some other goons: Louis Chislett, Aris Sionakidis, Torben Sell, Wenxing Zhou. I have never ceased resenting you all for diluting my attention, but at least one of you has a boat, so I'm willing to call it quits.

I'd thought that I might get out of mentioning my family by keeping these acknowledgements fairly work-focused. Unfortunately this darn pandemic happened, and in doing the right and proper middle-class thing by running home to mummy and daddy in the countryside, I've allowed them and my (overnumerous) siblings an unfortunate back door into my professional life. I'm therefore obliged to say: thanks Mum, thanks Dad, I couldn't have done it without you. Ruth, Sarah, Luke: I may well have been able to do it without you, but it would have been substantially less fun. You're all great.

I never mention it, but I spent a bit of time in London during the later stages of my PhD. Those who I worked with in the summer of 2022 are well-loved by myself, but unfortunately none of that stuff made the thesis, so you'll have to remain satisfied by being well-loved anonymously. If you want me to say something nice about you, I'll put it in a WhatsApp. The second half of my stay in London, at the Alan Turing Institute don't you know, brought with it many welcome new faces including Lukas Franken, Susana Garcia, and Ezra Webb. It also brought back some very welcome old ones including Seb Hickman and Sara Masarone (again), which was an absolute treat.

Eventually I made it back most of the way to Edinburgh, stopping off in the village of Auchendinny, Midlothian, much to the bemusement of my supervisor Tim. Who, pretty much alone in the western world, hasn't been mentioned so far. Grab the hanky Tim, this bit's for you!

At various points throughout my PhD, external factors and/or internal fidgetiness have prompted me to up sticks and change direction at short notice. These changes of direction have included change of academic direction, change of city and change of time zones\footnote{Bizarrely, the latter and the former did not overlap.}. At each of these points, any reasonable supervisor would have been well within his rights to say ``Okay, but we spent quite a while doing thing A'', ``Okay, but it would be nice to be in the same city'', ``Okay, but how about you work on this set of problems that are more relevant to my interests and expertise?'', ``Okay, but with two months to go do you really think now's the time?''. To say that Tim's shown a fair bit of grace in allowing me to pursue my interests, needs, and fancies, would be doing it down just a touch. I have never throughout all of my forgetting and/or turning up late to meetings, redefining deadlines, bad jokes, scrapes with international law, exposure to potential defunding, or any more of an indefinite list of things, felt that Tim was anywhere other than firmly and encouragingly behind me. I hope it's not going to my head to say: I was Tim's first student, he was my first supervisor, and we just about worked it out as we went along together. From our first meeting in Edinburgh when he told me that it would never be my job to try and impress him\footnote{There at least I may have succeeded as a student.}, to socially distant tins of cider outside the university library after the first quelling of lockdown, to a barbecue at sea in our last week as supervisor and supervisee, I owe Tim the immense privilege it's been to have had my life for the last four years.

On that triumphant note (and contradicting myself) I can't help but list some more of the people who've touched my life in the last four years, perhaps not academically, but who've made it an absolute pleasure to be around and about and in so doing have made the world of difference: Lydie T, Naomi CS, Michael C, Esme B, Bailey B, Beth B, Kathryn W, Sula C, Elinor CA, Rowan \st{B}H, Paul NJ, Hannah W, Vessela I, Conor C, Rob J, Eve D, Tanmay S, Freddie B, David N, Lizzie M, Lizzie J, George P, Mark A, Benji T, Daisy T, Molly S, Tommy S, Harrison F, Clem B, Lottie P, Tom D, Ollie S, Harry J, Clara D, Sophie O, Shereen S, Kai W, Hannah K, Frank D, Heather Y, Alex C, Alex M, Hannah S, Jo W, James F, Rosa H, James T, Cara W, Alex S, Lawrence (Barry) S, André V, Hawo A, Dilini K, Millie Z, Antonios PD, Jack E, Jen M, John M.

It's bizarre after all that to end with a plug, but so I shall. The work in this thesis comes in part from a variety of publications and pre-publications. Chapter~1 is based very loosely on the book chapter: ``Dimensionality and structure in cancer genomics: a statistical learning perspective'' \citep{bradley_dimensionality_2020}. Chapter~2 is adapted from ``Hierarchical Bayesian modeling identifies key considerations in the design of loop-mediated isothermal amplification assays'' \citep{bradley_hierarchical_2023}, while Chapter~3 is adapted from ``Data-driven design of targeted gene panels for estimating immunotherapy biomarkers'' \citep{bradley_data-driven_2021, bradley_data-driven_2022}. Chapter~4 has no associated literature, as it was being written up until the week of submission.

\end{document}